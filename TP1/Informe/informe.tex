\documentclass[10pt, a4paper, spanish]{article}
\parindent = 15 pt
\parskip 6px
\usepackage[width=15.5cm, left=3cm, top=2.5cm, height= 24.5cm]{geometry}
\usepackage[spanish]{babel}
\usepackage{graphicx}
\usepackage{amsmath}
\usepackage{amsfonts}
\usepackage{amssymb}
\usepackage[utf8]{inputenc}
\usepackage{listings}
\usepackage{color}

\usepackage{subcaption}
\captionsetup{compatibility=false}

\definecolor{mygreen}{rgb}{0,0.6,0}
\definecolor{mygray}{rgb}{0.5,0.5,0.5}
\definecolor{mymauve}{rgb}{0.58,0,0.82}

\lstset{ %
  backgroundcolor=\color{white},   % choose the background color
  basicstyle=\footnotesize,        % size of fonts used for the code
  breaklines=true,                 % automatic line breaking only at whitespace
  captionpos=b,                    % sets the caption-position to bottom
  commentstyle=\color{mygreen},    % comment style
  escapeinside={\%*}{*)},          % if you want to add LaTeX within your code
  keywordstyle=\color{blue},       % keyword style
  stringstyle=\color{mymauve},     % string literal style
}


%\renewcommand{\headrulewidth}{0.4pt} 
%\renewcommand{\footrulewidth}{0.4pt}

\usepackage{fancyhdr}
\pagestyle{fancy}
\fancyhf{}
\fancyhead[RO,LE]{Trabajo práctico N$^\circ$ 1 - Redes Neuronales}
\lhead{Panarello - Uhrich - Villanueva} % texto izquierda de la cabecera
%\chead{TEXTO} % texto centro de la cabecera
%\rhead{\thepage} % número de página a la derecha
%\lfoot{TEXTO} % texto izquierda del pie
\cfoot{\thepage}
\renewcommand{\headrulewidth}{0.1pt} % grosor de la línea de la cabecera
\renewcommand{\footrulewidth}{0.1pt} % grosor de la línea del pie

\begin{document}


\section{Diagnóstico de cáncer de mamas}

\subsection{Ejemplos}

\subsubsection{Pruebas con distintos learning rates}

Parámetros elegidos fijos:

\begin{itemize}
\item beta = 5
\item mini\_batch\_size = 1
\item epochs = 1000
\item epsilon = 0.05
\item reg\_param = 0.0
\end{itemize}


\begin{figure}[h]	
	\begin{subfigure}[b]{0.5\textwidth}
		\includegraphics[width=\linewidth]{fig/trainingerror_lr0,001_eps0,05_regparam0,00_beta5_batch1.png}
	\end{subfigure}
	\begin{subfigure}[b]{0.5\textwidth}
		\includegraphics[width=\linewidth]{fig/valerror_lr0,001_eps0,05_regparam0,00_beta5_batch1.png}
	\end{subfigure}

	\caption{\textbf{learning rate: 0.001}}
\end{figure}

\begin{figure}[h]	
	\begin{subfigure}[b]{0.5\textwidth}
		\includegraphics[width=\linewidth]{fig/trainingerror_lr0,0025_eps0,05_regparam0,00_beta5_batch1.png}
	\end{subfigure}
	\begin{subfigure}[b]{0.5\textwidth}
		\includegraphics[width=\linewidth]{fig/valerror_lr0,0025_eps0,05_regparam0,00_beta5_batch1.png}
	\end{subfigure}

	\caption{\textbf{learning rate: 0.0025}}
\end{figure}


\begin{figure}[h]	
	\begin{subfigure}[b]{0.5\textwidth}
		\includegraphics[width=\linewidth]{fig/trainingerror_lr0,005_eps0,05_regparam0,00_beta5_batch1.png}
	\end{subfigure}
	\begin{subfigure}[b]{0.5\textwidth}
		\includegraphics[width=\linewidth]{fig/valerror_lr0,005_eps0,05_regparam0,00_beta5_batch1.png}
	\end{subfigure}

	\caption{\textbf{learning rate: 0.005}}
\end{figure}

\newpage
\subsection{Código fuente}

\subsubsection*{Carga de datos}

\lstinputlisting[language=Python]{../ej1_data_loader.py}

\subsubsection*{Preprocesamiento de datos}

\lstinputlisting[language=Python]{../preprocessor.py}

\subsubsection*{Modelo}

\lstinputlisting[language=Python]{../layer_model.py}

\subsubsection*{Funciones sigmoideas}

\lstinputlisting[language=Python]{../sigmoid.py}

\subsubsection*{Algoritmos}

\lstinputlisting[language=Python]{../feed_forward_solver.py}

\subsubsection*{Test}

\lstinputlisting[language=Python]{../test2.py}

\newpage
 
\noindent{\textbf{Eficiencia energética}} \\

\newpage 

\end{document}
