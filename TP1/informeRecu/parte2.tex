\newpage
\section{Eficiencia energética}

%%--------------------------------------------------------------------------------------------------------------------------------------------------

\subsection{Introducción}

%%--------------------------------------------------------------------------------------------------------------------------------------------------

\subsection{Modelo}

Se probaron varias arquitecturas de red, incluyendo una con nivel más de profundidad respecto a la del punto 1, pero finalmente nos quedamos con la de 3 niveles (una capa oculta).

{
"layers": [{
"type": "input",
"size": 8
}, {
"type": "sigmoid",
"size": 20,
"beta": 20
}, {
"type": "tanh",
"size": 2,
"beta": 20
}]
}


\begin{figure}[ht!]
	\centering
	\includegraphics[width=1\linewidth]{fig/parte2-modelo.jpg}
	\caption{Modelo de la red}
\end{figure}

Se colocaron a la salida dos unidades del tipo tanh, cada una representando uno de los valores de salida a aproximar (energía y lo otro). Se utilizó tanh (x) 
dado que aproxima bastante a una recta dentro del codiminio (-1,1). Se utilizó la activación tanh(z / beta).  El valor beta nos permite ajustar la pendiente de la curva. 
Pendientes mas bajas aceleran el entrenamiento.



%%--------------------------------------------------------------------------------------------------------------------------------------------------

\subsection{Implementación}

Overfitting: El otro problema que encaramos es que, cuando el error de entrenamiento absoluto llega a bajar a cierto valor (aproximadamente 0.06, dependiendo de los parámetros), 
el error de testeo comienza a incrementarse (llega a un mínimo de 0.1 aproximadamente) y nunca baja. Intentamos palear esto incrementando el parámetro de regularización, pero no resultó.

\subsubsection{Prepocesamiento de datos}

Los datos pertenecientes al dataset fueron normalizados mediante la siguiente fórmula: 

\begin{align*}
		\frac{x - \mu}{max(features) - min(features)} 
\end{align*}

donde $x$ perteneces a features, y $\mu$ es la media de los features. Esto se hizo para que todos los datos  del dataset queden en el rango (-1,1).

\subsubsection{Pseudocódigo}

%%--------------------------------------------------------------------------------------------------------------------------------------------------

\subsection{Ejecución}

\subsubsection{Modo de uso}

\textbf{Para entrenar las redes:}

\noindent\texttt{\scriptsize{python trainnet1.py -m ej1.lmodel -o redentrenada1.params -t 3000 -e 0.01 -l 0.005 -b 1}} \\

\texttt{\scriptsize{-x tp1\_ej1\_training.csv}} \\

\noindent\texttt{\scriptsize{python predict2.py -m ej2.lmodel -p redentrenada2.params -x tp1\_ej2\_testing.csv}} \\





\subsubsection{Requerimientos}

%%--------------------------------------------------------------------------------------------------------------------------------------------------

\subsection{Resultados}

Probando combinaciones de $\beta$, learning rate y el término de regularización, lo mejor que pudimos lograr fue:

\begin{itemize}
	\item Error absoluto promedio de testing: 0.033. Esto es, promediando las distancias euclidianas entre el vector de dos posiciones predicho y el verdadero. 
	\item Error relativo promedio: 0.38 (38$\%$). Esto es la desviación relativa promedio de los valores verdaderos, tomado como el error absoluto / error promedio.
\end{itemize}

Como podemos ver, la estimación hasta el momento es bastante mala.

%%--------------------------------------------------------------------------------------------------------------------------------------------------

\subsection{Conclusiones}
